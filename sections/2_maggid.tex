\siman{
מַגִּיד
}

\instruction{
מגלה את המצות, מגביה את הקערה ואומר בקול רם:
}{Uncover the matsos, lift up the plate, and say in full voice:}

{\bfseries
הָא לַחְמָא עַנְיָא דִי אֲכָלוּ אַבְהָתָנָא בְּאַרְעָא דְמִצְרָיִם. כָּל דִכְפִין יֵיתֵי וְיֵיכֹל, כָּל דִצְרִיךְ יֵיתֵי וְיִפְסַח. הָשַׁתָּא הָכָא, לְשָׁנָה הַבָּאָה בְּאַרְעָא דְיִשְׂרָאֵל. הָשַׁתָּא עַבְדֵי, לְשָׁנָה הַבָּאָה בְּנֵי חוֹרִין.
}

\begin{english}
This is the bread of poverty that our ancestors ate in the land of Egypt. Let all who are hungry come and eat, let all who are needy come and make pesah. Let those who are here now be in the land of Yisroel in the coming year. Let those who are slaves now be free persons in the coming year.
\end{english}

\vspace{1em}

%%%%% FOUR QUESTIONS

\instruction{
מסיר הקערה מעל השלחן, מוזגין כוס שני וכאן הבן או אחד מן המסבים שואל:
}{Remove the matso plate from the table, pour the second cup of wine. The youngest person present asks:}

מַה נִּשְּׁתַּנָה הַלַּיְלָה הַזֶּה מִכָּל הַלֵּילוֹת? 
שֶׁבְּכָל הַלֵּילוֹת אָנוּ אוֹכְלִין חָמֵץ וּמַצָּה. הַלַּיְלָה הַזֶּה כֻּלּוֹ מַצָּה!
שֶׁבְּכָל הַלֵּילוֹת אָנוּ אוֹכְלִין שְׁאָר יְרָקוֹת. הַלַּיְלָה הַזֶּה מָרוֹר!
שֶׁבְּכָל הַלֵּילוֹת אֵין אָנוּ מַטְבִּילִין אֲפִילוּ פַּעַם אֶחָת. הַלַּיְלָה הַזֶּה שְׁתֵּי פְעָמִים!
שֶׁבְּכָל הַלֵּילוֹת אָנוּ אוֹכְלִין בֵּין יוֹשְׁבִין וּבֵין מְסֻבִּין. הַלַּיְלָה הַזֶּה כֻּלָּנוּ מְסֻבִּין!

\begin{english}
What makes this night different than all other nights, such that on all other nights, we eat leavened or unleavened bread but tonight, only unleavened? That on all other nights, we eat whatever vegetables we please but tonight, bitter ones? That on all other nights, we do not dip our foods even once, but tonight, twice? That on all other nights, we eat either sitting or reclining, but tonight, reclining?
\end{english}

\vspace{1em}

\instruction{
מניח את הקערה על השלחן. המצות תהיינה מגלות בשעת אמירת ההגדה.
}{Return the matso plate to the table. The matso should be partially uncovered as the haggadah is read.}

\break

עֲבָדִים הָיִינוּ לְפַרְעֹה בְּמִצְרָיִם, וַיּוֹצִיאֵנוּ יי אֱלֹהֵינוּ מִשָּׁם בְּיָד חֲזָקָה וּבִזְרוֹעַ נְטוּיָה. וְאִלּוּ לֹא הוֹצִיא הַקָּדוֹשׁ בָּרוּךְ הוּא אֶת אֲבוֹתֵינוּ מִמִּצְרָיִם, הֲרֵי אָנוּ וּבָנֵינוּ וּבְנֵי בָנֵינוּ מְשֻׁעְבָּדִים הָיִינוּ לְפַרְעֹה בְּמִצְרָיִם. וַאֲפִילוּ כֻּלָּנוּ חֲכָמִים, כֻּלָּנוּ נְבוֹנִים, כֻּלָנוּ זְקֵנִים, כֻּלָנוּ יוֹדְעִים אֶת הַתּוֹרָה, מִצְוָה עָלֵינוּ לְסַפֵּר בִּיצִיאַת מִצְרַיִם. וְכָל הַמַּרְבֶּה לְסַפֵּר בִּיצִיאַת מִצְרַיִם הֲרֵי זֶה מְשֻׁבָּח.

\begin{english}
We were slaves to Pharaoh in Egypt, and Hashem our God brought us out from there with a strong hand and an outstretched arm! If the Holy Blessed One had not brought our ancestors out of Egypt, we and our children and our children's children would be be enslaved to Pharaoh in Egypt. Even if we were all sages, all understanding, all distinguished, all full of Torah, the mitsvah would be upon us to tell about bring brought out of Egypt. Whoever multiplies this recounting of our exodus is praiseworthy.
\end{english}

\vspace{1em}

מַעֲשֶׂה בְּרַבִּי אֱלִיעֶזֶר וְרַבִּי יְהוֹשֻעַ וְרַבִּי אֶלְעָזָר בֶּן עֲזַרְיָה וְרַבְּי עֲקִיבָא וְרַבִּי טַרְפוֹן שֶהָיוּ מְסֻבִּין בִּבְנֵי בְרַק, וְהָיוּ מְסַפְּרִים בִּיצִיאַת מִצְרַיִם כָּל אוֹתוֹ הַלַּיְלָה עַד שֶׁבָּאוּ תַלְמִידֵיהֶם וְאָמְרוּ לָהֶם: רַבּוֹתֵינוּ, הִגִּיעַ זְמַן קְרִיאַת שְׁמַע שֶׁל שַׁחֲרִית.

\begin{english}
It happened that Rabbi Eliezar, Rabbi Yehoshua ben Azaria, Rabbi Akiva, and Rabbi Tarfon were reclining in B'nai Brak, and they recounted the exodus all night long until their students came and said to them, ``Teachers, it is time to recite the morning shema."
\end{english}

\vspace{1em}

אָמַר רַבִּי אֶלְעָזָר בֶּן עֲזַרְיָה: הֲרֵי אֲנִי כְבֶן שִׁבְעִים שָׁנָה, וְלֹא זָכִיתִי שֶׁתֵּאָמֵר יְצִיאַת מִצְרַיִם בַּלֵּילוֹת עַד שֶׁדְּרָשָׁהּ בֶּן זוֹמָא: שֶׁנֶּאֱמַר,
\pasuk{
לְמַעַן תִּזְכֹּר אֶת יוֹם צֵאתְךָ מֵאֶרֶץ מִצְרַיִם כֹּל יְמֵי חַיֶּיךָ.
}\glossed{
יְמֵי חַיֶּיךָ
}
הַיָמִים,\glossed{
כָּל יְמֵי חַיֶּיךָ
}
הַלֵּילוֹת. וַחֲכָמִים אוֹמְרִים: \glossed{
יְמֵי חַיֶּיךָ
}
הָעוֹלָם הַזֶּה, \glossed{
כֹּל יְמֵי חַיֶּיךָ 
}
לְהָבִיא לִימוֹת הַמָשִׁיחַ.

\begin{english}
Rabbi Elazar ben Azaria said, ``I am seventyish years old and I have not merited to understand why we recount the exodus at night, until I heard this drash of Ben Zoma: As it is said, \bibverse{So that you will remember the day you went out of the land of Egypt all the days of your life.} \engloss{The days of your life} the days. \engloss{All the days of your life} the nights." But the sages say: \engloss{The days of your life} this world. \engloss{All the days of your life} to include the messianic age.
\end{english}

\begin{center}
\begin{tabular}{c c c c}
{\small \instructionfont
יו"ד
} & {\small \instructionfont
ה"י
} & {\small \instructionfont
וי"ו
} & {\small \instructionfont
ה"י 
} \\
בָּרוּךְ הַמָּקוֹם, 
& בָּרוּךְ הוּא. 
& בָּרוּךְ שֶׁנָּתַן תּוֹרָה לְעַמּוֹ יִשְׂרָאֵל, 
& בָּרוּךְ הוּא. \\
\end{tabular}

\begin{english}
\vspace{0.5em}
Blessed is God, who is blessed! 

Blessed is the one who gave Torah to the people Yisroel, 

who is blessed.
\end{english}
\end{center}

\break

\vspace{1em}

כְּנֶגֶד אַרְבָּעָה בָנִים דִּבְּרָה תּוֹרָה. אֶחָד חָכָם, וְאֶחָד רָשָׁע, וְאֶחָד תָּם, וְאֶחָד שֶׁאֵינוֹ יוֹדֵעַ לִשְׁאוֹל.

\begin{english}
The Torah speaks in a manner appropriate for four children: one wise, one wicked, one simple, and one does not know how to ask.
\end{english}

\vspace{1em}

חָכָם מָה הוּא אוֹמֵר? מַה הָעֵדוֹת וְהַחֻקִּים וְהַמִשְׁפָּטִים אֲשֶׁר צִוָּה יי אֱלֹהֵינוּ אֶתְכֶם? וְאַף אַתָּה אֱמָר לוֹ כְּהִלְכוֹת הַפֶּסַח: אֵין מַפְטִירִין אַחַר הַפֶּסַח אֲפִיקוֹמָן.

\begin{english}
What does the wise one ask? ``What are these statues and these ordinances and laws that Hashem our God has commanded us?" Teach the child all the laws of pesah, up to: ``we do not eat a desert after the meat of the pesah."
\end{english}

\vspace{1em}

רָשָׁע מָה הוּא אוֹמֵר? מָה הָעֲבֹדָה הַזֹּאת לָכֶם? לָכֶם , וְלֹא לוֹ. וּלְפִי שֶׁהוֹצִיא אֶת עַצְמוֹ מִן הַכְּלָל כָּפַר בְּעִקָּר. וְאַף אַתָּה הַקְהֵה אֶת שִנָּיו וֶאֱמֹר לוֹ: בַּעֲבוּר זֶה עָשָׂה יי לִי בְּצֵאתִי מִמִּצְרָיִם. לִי - וְלֹא לוֹ. אִילּוּ הָיָה שָׁם, לֹא הָיָה נִגְאָל.

\begin{english}
What does the wicked one ask? ``What is this service for you?" `For you' $-$ as though to exclude themselves from the whole, they uproot a core principle. Blunt the child's teeth by saying, ``This is for the sake of what Hashem did for me when I went out of Egypt." `For me' $-$ as though if the child has been there, they would not have been redeemed.
\end{english}

\vspace{1em}

תָּם מָה הוּא אוֹמֵר? מַה זֹּאת? וְאָמַרְתָּ אֵלָיו: בְּחֹזֶק יָד הוֹצִיאָנוּ יי מִמִּצְרָיִם, מִבֵּית עֲבָדִים.

\begin{english}
What does the simple child ask? ``What is this?" Say to them, ``With a strong hand Hashem brought us out of Egypt, from the house of servitude."
\end{english}

\vspace{1em}

וְשֶׁאֵינוֹ יוֹדֵעַ לִשְׁאוֹל - אַתְּ פְּתַח לוֹ, שֶׁנֶּאֱמַר:
\pasuk{וְהִגַּדְתָּ לְבִנְךָ בַּיוֹם הַהוּא לֵאמֹר, בַּעֲבוּר זֶה עָשָׂה יי לִי בְּצֵאתִי מִמִּצְרָיִם.}

\begin{english}
The one who does not know what to ask, you must open up their mouth, as it says,``\bibverse{And you will tell your child on that day, saying, this is all on account of what Hashem did for me by taking me out of Egypt}." 
\end{english}

יָכוֹל מֵרֹאשׁ חֹדֶשׁ, תַּלְמוּד לוֹמַר 
\glossed{בַּיוֹם הַהוּא,}
 אִי בַּיוֹם הַהוּא יָכוֹל מִבְּעוֹד יוֹם, תַּלְמוּד לוֹמַר בַּעֲבוּר זֶה - בַּעֲבוּר זֶה לֹא אָמַרְתִּי אֶלָא בְּשָׁעָה שֶׁיֵּשׁ מַצָּה וּמָרוֹר מֻנָּחִים לְפָנֶיךָ.

\begin{english}
(Why could one not do so any time after Rosh Hodesh Nisan? Scripture says, \bibverse{on that day}. Why not during the day? Scripture says, \bibverse{on account of [what Hashem did for me]}, that is, ``I did not say anything except at the hour when there is matso and maror set before you.")
\end{english}

\break

\vspace{1em}

מִתְּחִלָּה עוֹבְדֵי עֲבוֹדָה זָרָה הָיוּ אֲבוֹתֵינוּ, וְעַכְשָׁיו קֵרְבָנוּ הַמָּקוֹם לַעֲבֹדָתוֹ, שֶׁנֶּאֱמַר:
\pasuk{
וַיֹאמֶר יְהוֹשֻעַ אֶל כָּל הָעָם, כֹּה אָמַר יי אֱלֹהֵי יִשְׂרָאֵל: בְּעֵבֶר הַנָּהָר יָשְׁבוּ אֲבוֹתֵיכֶם מֵעוֹלָם, תֶּרַח אֲבִי אַבְרָהָם וַאֲבִי נָחוֹר, וַיַּעַבְדוּ אֱלֹהִים אֲחֵרִים. וָאֶקַּח אֶת אֲבִיכֶם אֶת אַבְרָהָם מֵעֵבֶר הַנָּהָר וָאוֹלֵךְ אוֹתוֹ בְּכָל אֶרֶץ כְּנָעַן, וָאַרְבֶּה אֶת זַרְעוֹ וָאֶתֵּן לוֹ אֶת יִצְחָק, וָאֶתֵּן לְיִצְחָק אֶת יַעֲקֹב וְאֶת עֵשָׂו. וָאֶתֵּן לְעֵשָׂו אֶת הַר שֵּׂעִיר לָרֶשֶׁת אֹתוֹ, וְיַעֲקֹב וּבָנָיו יָרְדוּ מִצְרָיִם.
}

\begin{english}
At first, our ancestors were idolators, and now God has brought us close to God's service, as it says, \bibverse{And Yehoshua said to the whole people, so says Hashem, God of Israel: on the other side of the river, our ancestors lived from time immemorial: Terah, the father of Avraham and Nahor, and they served other gods. But I took your father, Avraham, from the other side of the river and I walked him through the whole land of Canaan. I multiplied his descendants and I gave him Yitzhak, and I gave Yakov and Esav to Yitzhak. And I gave to Esav the mountain of Seir as an inheritance, but Yakov went down to Egypt.}
\end{english}

\vspace{2em}

בָּרוּךְ שׁוֹמֵר הַבְטָחָתוֹ לְיִשְׂרָאֵל, בָּרוּךְ הוּא. שֶׁהַקָּדוֹשׁ בָּרוּךְ הוּא חִשַּׁב אֶת הַקֵּץ, לַעֲשׂוֹת כְּמַה שֶּׁאָמַר לְאַבְרָהָם אָבִינוּ בִּבְרִית בֵּין הַבְּתָרִים, שֶׁנֶּאֱמַר: 
\pasuk{
וַיֹּאמֶר 
\penalty-1000
לְאַבְרָם, יָדֹע תֵּדַע כִּי גֵר יִהְיֶה זַרְעֲךָ בְּאֶרֶץ לֹא לָהֶם, וַעֲבָדוּם וְעִנּוּ אֹתָם אַרְבַּע מֵאוֹת שָׁנָה. וְגַם אֶת הַגּוֹי אֲשֶׁר יַעֲבֹדוּ דָּן אָנֹכִי וְאַחֲרֵי כֵן יֵצְאוּ בִּרְכֻשׁ גָּדוֹל.
}

\begin{english}
Blessed is the one who kept trust with Yisroel, blessed is God that the Holy Blessed One saw the matter through to the end, and did what God had told Avraham at the bris bein habasarim, as it says, \bibverse{And God said to Avrom, know that your seed with be alien in a land they do not possess, and that they will be enslaved and oppressed for four hundred years. But I will judge that nation that they serve, and after that they will go out with great wealth.}
\end{english}

\instruction{
מכסה את המצות ומגביה את הכוס.
}{Cover the matsos, and lift up the cup.}

וְהִיא שֶׁעָמְדָה לַאֲבוֹתֵינוּ וְלָנוּ! שֶׁלֹּא אֶחָד בִּלְבָד עָמַד עָלֵינוּ לְכַלּוֹתֵנוּ, אֶלָּא שֶׁבְּכָל דּוֹר וָדוֹר עוֹמְדִים עָלֵינוּ לְכַלּוֹתֵנוּ, וְהַקָּדוֹשׁ בָּרוּךְ הוּא מַצִּילֵנוּ מִיָּדָם.

\begin{english}
It stood for our ancestors and for us! And not just once one time did someone stand up to destroy us, but in each and every generation our enemies have stood up to destroy us, and the Holy Blessed One has delivered us from their hands.
\end{english}

\instruction{
יניח הכוס מידו ויגלה את המצות.
}{Put the cup down and uncover the matsos.}

\break

\begin{center}
{\large \bfseries \textcolor{light-gray}{\pasuk{
אֲרַמִּי אֹבֵד אָבִי...
}}}
\end{center}

%%%%% FIRST PASUK OF ARAMI OVED AVI

צֵא וּלְמַד מַה בִּקֵּש לָבָן הָאֲרַמִי לַעֲשׂוֹת לְיַעֲקֹב אָבִינוּ. שֶׁפַּרְעֹה לֹא גָזַר אֶלָּא עַל הַזְּכָרִים וְלָבָן בִּקֵּשׁ לַעֲקוֹר אֶת הַכֹּל, שֶׁנֶּאֱמַר: 
\pasuk{
אֲרַמִּי אֹבֵד אָבִי, וַיֵּרֶד מִצְרַיְמָה וַיָּגָר שָׁם בִּמְתֵי מְעָט, וַיְהִי שָׁם לְגוֹי גָּדוֹל, עָצוּם וָרָב.
}\glossed{
וַיֵרֶד מִצְרַיְמָה
}
אָנוּס עַל פִּי הַדִּבּוּר.\glossed{
וַיָּגָר שָׁם
}
מְלַמֵּד שֶׁלֹא יָרַד יַעֲקֹב אָבִינוּ לְהִשְׁתַּקֵּעַ בְּמִצְרַיִם אֶלָּא לָגוּר שָׁם, שֶׁנֶּאֱמַר: 
\pasuk{
וַיֹּאמְרוּ אֶל פַּרְעֹה, לָגוּר בָּאָרֶץ בָּאנוּ, כִּי אֵין מִרְעֶה לַצֹּאן אֲשֶׁר לַעֲבָדֶיךָ, כִּי כָבֵד הָרָעָב בְּאֶרֶץ כְּנָעַן. וְעַתָּה יֵשְׁבוּ נָא עֲבָדֶיךָ בְּאֶרֶץ גֹּשֶן.
}\glossed{
בִּמְתֵי מְעָט
}
כְּמַה שֶּׁנֶּאֱמַר: 
\pasuk{
בְּשִׁבְעִים נֶפֶשׁ יָרְדוּ אֲבוֹתֶיךָ מִצְרָיְמָה, וְעַתָּה שָׂמְךָ יי אֱלֹהֶיךָ כְּכוֹכְבֵי הַשָּׁמַיִם לָרֹב.
}\glossed{
וַיְהִי שָׁם לְגוֹי
}
מְלַמֵּד שֶׁהָיוּ יִשְׂרָאֵל מְצֻיָּנִים שָׁם.\glossed{
גָּדוֹל, עָצוּם
}
כְּמה שֶּׁנֶּאֱמַר: 
\pasuk{
וּבְנֵי יִשְׂרָאֵל פָּרוּ וַיִּשְׁרְצוּ וַיִּרְבּוּ וַיַּעַצְמוּ בִּמְאֹד מְאֹד, וַתִּמָּלֵא הָאָרֶץ אֹתָם.
}\glossed{
וָרָב
}
כְּמַה שֶּׁנֶּאֱמַר: 
\pasuk{
רְבָבָה כְּצֶמַח הַשָּׂדֶה נְתַתִּיךְ, וַתִּרְבִּי וַתִּגְדְּלִי וַתָּבֹאִי בַּעֲדִי עֲדָיִים, שָׁדַיִם נָכֹנוּ וּשְׂעָרֵךְ צִמֵּחַ, וְאַתְּ עֵרֹם וְעֶרְיָה. וָאֶעֱבֹר עָלַיִךְ וָאֶרְאֵךְ מִתְבּוֹסֶסֶת בְּדָמָיִךְ, וָאֹמַר לָךְ בְּדָמַיִךְ חֲיִי, וָאֹמַר לָךְ בְּדָמַיִךְ חֲיִי.
}

\begin{english}
Go and learn what Lovan the Aramean meant to do to Yakov avinu! Pharaoh only decreed destruction for the male children, but Lovan wished to destroy all of them, as it says: \bibverse{An Aramean tried to destroy my father [alternatively: my father was a wandering Aramean] and he went down to Egypt and sojourned there in meager numbers, and became a great nation there, vast and numerous.} \engloss{And he went down to Egypt} he was forced to do so. \engloss{And sojourned there} this teaches that Yakov avinu did not go down to settle in Egypt, but to sojourn there, as it says, \bibverse{Say to Pharaoh, we have come to sojourn in the land, as your servant has nothing to pasture his flocks with, since the famine is great in the land. Now, please grant your servant the land of Goshen.} \engloss{In meager numbers} as it says, \bibverse{With seventy souls in all our ancestors went down to Egypt, and now Hashem your God has given you numbers like the stars in the heavens.} \engloss{And became a [great] nation} this teaches that Yisroelites were a distinct people there. \engloss{Vast} as it says, \bibverse{And the Yisroelites were fruitful like rats and grew numerous and large beyond measure, and they filled up the very land.} \engloss{And numerous} as it says, \bibverse{I gave you growth like the plants of the fields, and you increased and matured and became beautiful. Your breasts were formed and your hair grown, but you were naked and bare. ``And I passed over you and saw you wallowing in your blood, and I said to you, `In your blood, live!'" and he said to you, `In your blood, live!'}
\end{english}

\break
%%%%% SECOND PASUK OF ARAMI OVED AVI
\begin{center}
{\large \bfseries \textcolor{light-gray}{\pasuk{
וַיָּרֵעוּ אֹתָנוּ הַמִּצְרִים וַיְעַנּוּנוּ...
}}}
\end{center}

\ \pasuk{
וַיָּרֵעוּ אֹתָנוּ הַמִּצְרִים וַיְעַנּוּנוּ, וַיִתְּנוּ עָלֵינוּ עֲבֹדָה קָשָׁה.
}\glossed{
וַיָּרֵעוּ אֹתָנוּ הַמִּצְרִים 
}
כְּמָה שֶׁנֶּאֱמַר: 
\pasuk{
הָבָה נִתְחַכְמָה לוֹ פֶּן יִרְבֶּה, וְהָיָה כִּי תִקְרֶאנָה מִלְחָמָה וְנוֹסַף גַם הוּא עַל שׂנְאֵינוּ וְנִלְחַם בָּנוּ, וְעָלָה מִן הָאָרֶץ.
}\glossed{
וַיְעַנּוּנוּ
}
כְּמָה שֶׁנֶּאֱמַר: 
\pasuk{
וַיָּשִׂימוּ עָלָיו שָׂרֵי מִסִּים לְמַעַן עַנֹּתוֹ בְּסִבְלֹתָם. וַיִּבֶן עָרֵי מִסְכְּנוֹת לְפַרְעֹה. אֶת פִּתֹם וְאֶת רַעַמְסֵס.
}\glossed{
וַיִתְּנוּ עָלֵינוּ עֲבֹדָה קָשָׁה
}
כְּמָה שֶׁנֶּאֱמַר: 
\pasuk{
וַיַעֲבִדוּ מִצְרַיִם אֶת בְּנֵי יִשְׂרָאֵל בְּפָרֶךְ.
}

\begin{english}
\bibverse{And the Egyptions were evil to us and they oppressed us, and they gave us hard labor.} \engloss{And the Egyptians were evil to us} as it says, \bibverse{Let us concoct a scheme against this people, lest it multiply, and when war is declared it will join with our enemies to make war on us, rise up out from the land.} \engloss{And they oppressed us} as it says, \bibverse{They set over the people overseers to oppress them with burdens. And they built for Pharaoh store-cities: Pithom and Raamses.} \engloss{And they set over them hard labor} as it says, \bibverse{And the Egyptians enslaved the Yisroelites with crushing work.}
\end{english}


%%%%% THIRD PASUK OF ARAMI OVED AVI
\begin{center}
{\large \bfseries \textcolor{light-gray}{\pasuk{
וַנִּצְעַק אֶל יי אֱלֹהֵי אֲבֹתֵינוּ...
}}}
\end{center}

\ \pasuk{
וַנִּצְעַק אֶל יי אֱלֹהֵי אֲבֹתֵינוּ, וַיִּשְׁמַע יי אֶת קֹלֵנוּ, וַיַּרְא אֶת עָנְיֵנוּ וְאֶת עֲמָלֵנוּ וְאֶת לַחֲצֵנוּ.
}\glossed{
וַנִּצְעַק אֶל יי אֱלֹהֵי אֲבֹתֵינוּ
}
כְּמָה שֶׁנֶּאֱמַר:
\pasuk{
וַיְהִי בַיָּמִים הָרַבִּים הָהֵם וַיָּמָת מֶלֶךְ מִצְרַיִם, וַיֵּאָנְחוּ בְנֵי יִשְׂרָאֵל מִן הָעֲבוֹדָה וַיִּזְעָקוּ, וַתַּעַל שַׁוְעָתָם אֶל הָאֱ-לֹהִים מִן הָעֲבֹדָה.
}\glossed{
וַיִּשְׁמַע יי אֶת קֹלֵנוּ
}
כְּמָה שֶׁנֶּאֱמַר: 
\pasuk{
וַיִּשְׁמַע אֱלֹהִים אֶת נַאֲקָתָם, וַיִּזְכּוֹר אֱלֹהִים אֶת בְּרִיתוֹ אֶת אַבְרָהָם, אֶת יִצְחָק וְאֶת יַעֲקֹב.
}\glossed{
וַיַּרְא אֶת עָנְיֵנוּ
}
זוֹ פְּרִישׁוּת דֶּרֶךְ אֶרֶץ, כְּמָה שֶׁנֶּאֱמַר: 
\pasuk{
וַיַּרְא אֱלֹהִים אֶת בְּני יִשְׂרָאֵל וַיֵּדַע אֱלֹהִים.
}\glossed{
וְאֶת עֲמָלֵנוּ
}
אֵלּוּ הַבָּנִים. כְּמָה שֶׁנֶּאֱמַר: 
\pasuk{
כָּל הַבֵּן הַיִּלּוֹד הַיְאֹרָה תַּשְׁלִיכֻהוּ וְכָל הַבַּת תְּחַיּוּן.
}\glossed{
וְאֶת לַחֲצֵנוּ
} 
זֶה הַדְּחַק, כְּמָה שֶׁנֶּאֱמַר: 
\pasuk{
וְגַם רָאִיתִי אֶת הַלַּחַץ אֲשֶׁר מִצְרַיִם לֹחֲצִים אֹתָם.
}

\begin{english}
\bibverse{And we cried out to Hashem, God of our ancestors, and Hashem listened to our voice, and saw our affliction and our labors and our oppression.} \engloss{And we cried out to Hashem, God of our ancestors} as it says, \bibverse{And after many days, the king of Egypt died, and the Yisroelites sighed from their labor and cried out, and their shout arose from their labors and came before God.} \engloss{And Hashem listened to our voice} as it says, \bibverse{And God listened to their groaning, and God remembered God's covenant with Avraham, with Yitzhak, and with Yakov.} \engloss{And saw our affliction} this is separation from intimacy, as it says, \bibverse{And God saw the Yisroelites and took notice.} \engloss{And our labors} these are the children, as it says, \bibverse{Every son that is born, you should throw him into the river, but the daughters may live.} \engloss{And our oppression} this is the violence, as it says, \bibverse{I have also seen the oppression by which the Egyptians oppress them.}
\end{english}

\break

%%%%% FOURTH PASUK OF ARAMI OVED AVI
\begin{center}
{\large \bfseries \textcolor{light-gray}{\pasuk{
וַיּוֹצִאֵנוּ יי מִמִצְרַיִם בְּיָד חֲזָקָה וּבִזְרֹעַ נְטוּיָה...
}}}
\end{center}

\ \pasuk{
וַיּוֹצִאֵנוּ יי מִמִצְרַיִם בְּיָד חֲזָקָה וּבִזְרֹעַ נְטוּיָה, וּבְמֹרָא גָּדֹל, וּבְאֹתוֹת וּבְמֹפְתִים.
}\glossed{
וַיּוֹצִאֵנוּ יי מִמִצְרַיִם
}
לֹא עַל יְדֵי מַלְאָךְ, וְלֹא עַל יְדֵי שָׂרָף, וְלֹא עַל יְדֵי שָׁלִיחַ, אֶלָּא הַקָּדוֹשׁ בָּרוּךְ הוּא בִּכְבוֹדוֹ וּבְעַצְמוֹ, שֶׁנֶּאֱמַר:

\begin{english}
\bibverse{And Hashem brought us out of Egypt with a strong hand and an outstretched arm, with great terror, and with signs and portents.} \engloss{And Hashem brought us out of Egypt} not by an angel, nor by a fiery creature, nor by a messenger, but by the Holy Blessed One in all of the divine glory, personally, as it says, 
\end{english}

\begin{myquote}

\pasuk{
וְעָבַרְתִּי בְאֶרֶץ מִצְרַיִם בַּלַּיְלָה הַזֶּה, וְהִכֵּיתִי כָּל בְּכוֹר בְּאֶרֶץ מִצְרַיִם מֵאָדָם וְעַד בְּהֵמָה, וּבְכָל אֱלֹהֵי מִצְרַיִם אֶעֱשֶׂה שְׁפָטִים. אֲנִי יי.
}\glossed{
וְעָבַרְתִּי בְאֶרֶץ מִצְרַיִם בַּלַּיְלָה הַזֶּה
}
אֲנִי וְלֹא מַלְאָךְ, \glossed{
וְהִכֵּיתִי כָּל בְכוֹר בְּאֶרֶץ מִצְרַיִם
}
אֲנִי וְלֹא שָׂרָף, \glossed{
וּבְכָל אֱלֹהֵי מִצְרַיִם אֶעֱשֶׂה שְׁפָטִים
}
אֲנִי ולֹא הַשָּׁלִיחַ. \glossed{
אֲנִי יי
}
אֲנִי הוּא ולֹא אַחֵר. 

\begin{english}
\bibverse{And I will pass over Egypt on this night, and I will strike every firstborn in the land of Egypt, from the humans to the animals, and I will judge all of the gods of Egypt. I am Hashem.} \engloss{And I will pass over Egypt on this night} I, and no angel. \engloss{And I will strike every firstborn in the land of Egypt} I, and no fiery creature. \engloss{And I will judge all the gods of Egypt} I, and no messenger. \engloss{I am Hashem} I, and no other.
\end{english}
\end{myquote}

\ \glossed{
בְּיָד חֲזָקָה
}
זוֹ הַדֶּבֶר, כְּמָה שֶׁנֶּאֱמַר: 
\pasuk{
הִנֵּה יַד יי הוֹיָה בְּמִקְנְךָ אֲשֶׁר בַּשָּׂדֶה, בַּסּוּסִים, בַּחֲמֹרִים, בַּגְּמַלִּים, בַּבָּקָר וּבַצֹּאן, דֶבֶר כָּבֵד מְאֹד.
}\glossed{
וּבִזְרֹעַ נְטוּיָה
}
זוֹ הַחֶרֶב, כְּמָה שֶׁנֶּאֱמַר: 
\pasuk{
וְחַרְבּוֹ שְׁלוּפָה בְּיָדוֹ, נְטוּיָה עַל יְרוּשָלַיִם.
}\glossed{
וּבְמֹרָא גָּדֹל
}
זוֹ גִלּוּי שְׁכִינָה, כְּמָה שֶׁנֶּאֱמַר: 
\pasuk{
אוֹ הֲנִסָּה אֱלֹהִים לָבֹא לָקַחַת לוֹ גוֹי מִקֶרֶב גּוֹי בְּמַסֹּת בְּאֹתֹת וּבְמוֹפְתִים, וּבְמִלְחָמָה וּבְיָד חֲזָקָה וּבִזְרוֹעַ נְטוּיָה, וּבְמוֹרָאִים גְּדֹלִים, כְּכֹל אֲשֶׁר עָשָׂה לָכֶם יי אֱלֹהֵיכֶם בְּמִצְרַיִם לְעֵינֶיךָ.
}\glossed{
וּבְאֹתוֹת
}
זֶה הַמַּטֶּה, כְּמָה שֶׁנֶּאֱמַר: 
\pasuk{
וְאֶת הַמַּטֶּה הַזֶּה תִּקַּח בְּיָדְךָ, אֲשֶׁר תַּעֲשֶׂה בּוֹ אֶת הָאֹתֹת.
}

\begin{english}
\engloss{With a strong hand} this is the animal disease, as it says, \bibverse{This is the hand of Hashem on the livestock that are in the field, on the horses, on the donkeys, on the camels, on the cattle and on the flocks, a very serious disease.} \engloss{And with an outstretched arm} this is the sword, as it says, \bibverse{with a drawn sword in hand, stretched out over Yerushalayim} \engloss{With great terror} this is the revelation of the Shekhinah, as it says, \bibverse{Or has God prepared to come and take a nation from the midst of another nation, with trials, with signs and portents, with warfare and a strong hand and outstretched arm, with great terrors, as everything which Hashem your God did for you in Egypt, in your sight.} \engloss{And with signs} this is the staff, as it says, \bibverse{And the staff which is in your hand, with which you shall work the signs---}
\end{english}

\instruction{
נוהגין להטיף טפה מן הכוס באמירת דם, ואש, ותימרות עשן, עשר המכות, דצ"ך, עד"ש, באח"ב, ביחד ט"ז פעם
}{It is customary to spill a drop from the cup upon saying, `blood,' and `fire,' and `pillars of smoke,' each of the ten plagues, ``d'tsakh," ``adash," ``b'ahav," together, sixteen times.}

\glossed{
וּבְמֹפְתִים
}
זֶה הַדָּם, כְּמָה שֶׁנֶּאֱמַר: 
\pasuk{
וְנָתַתִּי מוֹפְתִים בַּשָּׁמַיִם וּבָאָרֶץ, 
}

\begin{english}
\engloss{---and portents} this is the blood, as it says, \bibverse{And I will give portents in the heavens and on land:}
\end{english}

\makkah{\pasuk{
דָּם 
}}{blood}

\makkah{\pasuk{
וָאֵשׁ 
}}{and fire}

\makkah{\pasuk{
וְתִימְרוֹת עָשָׁן.
}}{and pillars of smoke}

%%%%% TEN PLAGUES

דָבָר אַחֵר: בְּיָד חֲזָקָה - שְׁתַּיִם, וּבִזְרֹעַ נְטוּיָה - שְׁתַּיִם, וּבְמֹרָא גָּדֹל - שְׁתַּיִם, וּבְאֹתוֹת - שְׁתַּיִם, וּבְמֹפְתִים - שְׁתַּיִם. אֵלּוּ עֶשֶׂר מַכּוֹת שֶׁהֵבִיא הַקָּדוֹשׁ בָּרוּךְ הוּא עַל הַמִּצְרִים בְּמִצְרַיִם, וְאֵלּוּ הֵן:

\begin{english}
Another interpretation: \engloss{With a strong hand} two [plagues]. \engloss{And an outstretched arm} two [plagues]. \engloss{With great terror} two [plagues]. \engloss{With signs} two [plagues]. \engloss{And portents} this is two [plagues]. These are the ten plagues, that the Holy Blessed One brought on Egypt:
\end{english}

\vspace{1em}

\makkah{
דָּם
}{blood}

\makkah{
צְפַרְדֵּעַ
}{frog}

\makkah{
כִּנִּים
}{lice}

\makkah{
עָרוֹב
}{wild beasts}

\makkah{
דֶּבֶר
}{animal disease}

\makkah{
שְׁחִין
}{boils}

\makkah{
בָּרָד
}{hail}

\makkah{
אַרְבֶּה
}{locusts}

\makkah{
חֹשֶׁךְ
}{darkness}

\makkah{
מַכַּת בְּכוֹרוֹת
}{slaying the firstborn}

רַבִּי יְהוּדָה הָיָה נוֹתֵן בָּהֶם סִמָּנִים: 

\begin{english}
Rabbi Yehuda would give them the following acronyms:
\end{english}

\begin{center}
{\Large \bfseries
דְּצַ"ךְ עַדַ"שׁ בְּאַחַ"ב.
}
\end{center}

\break

%%%%% MULTIPLYING PLAGUES

רַבִּי יוֹסֵי הַגְּלִילִי אוֹמֵר: מִנַּיִן אַתָּה אוֹמֵר שֶׁלָּקוּ הַמִּצְרִים בְּמִצְרַיִם עֶשֶׂר מַכּוֹת וְעַל הַיָּם לָקוּ חֲמִשִּׁים מַכּוֹת? בְּמִצְרַיִם מָה הוּא אוֹמֵר?
\pasuk{
וַיֹּאמְרוּ הַחַרְטֻמִּים אֶל פַּרְעֹה: אֶצְבַּע אֱלֹהִים הִוא, 
}
וְעַל הַיָּם מָה הוּא אוֹמֵר?
\pasuk{
וַיַּרְא יִשְׂרָאֵל אֶת הַיָד הַגְּדֹלָה אֲשֶׁר עָשָׂה יי בְּמִצְרַיִם, וַיִּירְאוּ הָעָם אֶת יי, וַיַּאֲמִינוּ בַּיְיָ וּבְמשֶׁה עַבְדוֹ.
}
כַּמָּה לָקוּ בְאֶצְבַּע? עֶשֶׂר מַכּוֹת. אֱמוֹר מֵעַתָּה: בְּמִצְרַיִם לָקוּ עֶשֶׂר מַכּוֹת וְעַל הַיָּם לָקוּ חֲמִשִּׁים מַכּוֹת.

\begin{english}
Rabbi Yosi the Gallilean says: How do you know that the Egyptians were struck with ten plagues in Egypt and at the sea, fifty plagues? In Egypt, what does it say? \bibverse{And Pharaoh's court sorcerers said, ``this is the finger of God!"} And at the sea, what does it say? \bibverse{And Yisroel saw the great hand with which Hashem dealt with Egypt, and the people feared Hashem, and they believed in Hashem and Moshe, the servant of God.} How much were they struck by the finger? Ten plagues. Say from this that the Egyptians were struck with ten plagues in Egypt and at the sea, fifty plagues.
\end{english}

\vspace{1em}

רַבִּי אֱלִיעֶזֶר אוֹמֵר: מִנַּיִן שֶׁכָּל מַכָּה וּמַכָּה שֶׁהֵבִיא הַקָּדוֹשׁ בָּרוּךְ הוּא עַל הַמִּצְרִים בְּמִצְרַיִם הָיְתָה שֶׁל אַרְבַּע מַכּוֹת? שֶׁנֶּאֱמַר:
\pasuk{
יְשַׁלַּח בָּם חֲרוֹן אַפּוֹ, עֶבְרָה וָזַעַם וְצָרָה, מִשְׁלַחַת מַלְאֲכֵי רָעִים. 
}\glossed{
עֶבְרָה
}
אַחַת,\glossed{
וָזַעַם
}
שְׁתַּיִם,\glossed{
וְצָרָה
}
שָׁלשׁ,\glossed{
מִשְׁלַחַת מַלְאֲכֵי רָעִים
}
אַרְבַּע. אֱמוֹר מֵעַתָּה: בְּמִצְרַיִם לָקוּ אַרְבָּעִים מַכּוֹת וְעַל הַיָּם לָקוּ מָאתַיִם מַכּוֹת.

\begin{english}
Rabbi Eliezar says: How do you know that each and every plague that the Holy Blessed One brought on the Egyptians in Egypt was four plagues? As it says, \bibverse{God dispatches upon them the heat of God's anger; wrath, indignation, and trouble, sending the angels of evil.} \engloss{Wrath} this is one. \engloss{Indignation} this is two. \engloss{And trouble} this is three. \engloss{Sending the angels of evil} this is four. Say from this that in Egypt they were struck by forty plagues and at the sea, two hundred plagues.
\end{english}

\vspace{1em}

רַבִּי עֲקִיבֶא אוֹמֵר: מִנַּיִן שֶׁכָּל מַכָּה וּמַכָּה שֶׁהֵבִיא הַקָּדוֹשׁ בָּרוּךְ הוּא עַל הַמִּצְרִים בְּמִצְרַיִם הָיְתָה שֶׁל חָמֵשׁ מַכּוֹת? שֶׁנֶּאֱמַר:
\pasuk{
יְשַׁלַּח בָּם חֲרוֹן אַפּוֹ, עֶבְרָה וָזַעַם וְצָרָה, מִשְׁלַחַת מַלְאֲכֵי רָעִים.
}\glossed{
חֲרוֹן אַפּוֹ
}
אַחַת,\glossed{
עֶבְרָה
}
שְׁתַּיִם,\glossed{
וָזַעַם
}
שָׁלֹשׁ,\glossed{
וְצָרָה
}
אַרְבַּע,\glossed{
מִשְׁלַחַת מַלְאֲכֵי רָעִים
}
חָמֵשׁ. אֱמוֹר מֵעַתָּה: בְּמִצְרַיִם לָקוּ חֲמִשִּׁים מַכּוֹת וְעַל הַיָּם לָקוּ חֲמִשִּׁים וּמָאתַיִם מַכּוֹת.

\begin{english}
Rabbi Akiva says: How do you know that each and every plague that the Holy Blessed One brought on the Egyptians in Egypt was five plagues? As it says, \bibverse{God dispatches upon them the heat of God's anger, wrath, indignation, and trouble, sending the angels of evil.} \engloss{Anger} this is one. \engloss{Wrath} this is two. \engloss{Indignation} this is three. \engloss{And trouble} this is four. \engloss{Sending the angels of evil} this is five. Say from this that in Egypt they were struck by fifty plagues and at the sea, two hundred and fifty plagues.
\end{english}

\begin{center}
{\bf
כַּמָּה מַעֲלוֹת טוֹבוֹת לַמָּקוֹם עָלֵינוּ!
}

\begin{english}
How inordinately good has God been to us! 
\end{english}

\vspace{1em}

%%%%%% DAYEINU

אִלּוּ הוֹצִיאָנוּ מִמִּצְרַיִם וְלֹא עָשָׂה בָהֶם שְׁפָטִים, דַּיֵּינוּ.

אִלּוּ עָשָׂה בָהֶם שְׁפָטִים, וְלֹא עָשָׂה בֵאלֹהֵיהֶם, דַּיֵּינוּ.

אִלּוּ עָשָׂה בֵאלֹהֵיהֶם, וְלֹא הָרַג אֶת בְּכוֹרֵיהֶם, דַּיֵּינוּ.

אִלּוּ הָרַג אֶת בְּכוֹרֵיהֶם וְלֹא נָתַן לָנוּ אֶת מָמוֹנָם, דַּיֵּינוּ.

אִלּוּ נָתַן לָנוּ אֶת מָמוֹנָם וְלֹא קָרַע לָנוּ אֶת הַיָּם, דַּיֵּינוּ.

אִלּוּ קָרַע לָנוּ אֶת הַיָּם וְלֹא הֶעֱבִירָנוּ בְתוֹכוֹ בֶּחָרָבָה, דַּיֵּינוּ.

אִלּוּ הֶעֱבִירָנוּ בְתוֹכוֹ בֶּחָרָבָה וְלֹא שִׁקַּע צָרֵנוּ בְּתוֹכוֹ, דַּיֵּינוּ.

אִלּוּ שִׁקַּע צָרֵנוּ בְּתוֹכוֹ וְלֹא סִפֵּק צָרְכֵּנוּ בַּמִּדְבָּר אַרְבָּעִים שָׁנָה, דַּיֵּינוּ.

אִלּוּ סִפֵּק צָרְכֵּנוּ בַּמִּדְבָּר אַרְבָּעִים שָׁנָה ולֹא הֶאֱכִילָנוּ אֶת הַמָּן, דַּיֵּינוּ.

אִלּוּ הֶאֱכִילָנוּ אֶת הַמָּן וְלֹא נָתַן לָנוּ אֶת הַשַׁבָּת, דַּיֵּינוּ.

אִלּוּ נָתַן לָנוּ אֶת הַשַׁבָּת, וְלֹא קֵרְבָנוּ לִפְנֵי הַר סִינַי, דַּיֵּינוּ.

אִלּוּ קֵרְבָנוּ לִפְנֵי הַר סִינַי, וְלֹא נָתַן לָנוּ אֶת הַתּוֹרָה, דַּיֵּינוּ.

אִלּוּ נָתַן לָנוּ אֶת הַתּוֹרָה וְלֹא הִכְנִיסָנוּ לְאֶרֶץ יִשְׂרָאֵל, דַּיֵינוּ.

אִלּוּ הִכְנִיסָנוּ לְאֶרֶץ יִשְׂרָאֵל וְלֹא בָנָה לָנוּ אֶת בֵּית הַבְּחִירָה, דַּיֵּינוּ.

\vspace{1em}

\begin{english} 
Had God brought us out of Egypt 

but not judged them, \hspace{2em} not judged their gods, \hspace{2em} not slain their firstborn, \hspace{2em} not split the sea for us, \hspace{3em} not brought us through on dry land, 

not drowned our oppressors, 

not seen to our needs for 40 years in the desert, \hspace{2em} not fed us mana, 

not given us Shabos, \hspace{1em} not come close to us at Sinai, \hspace{1em} not given us Torah, 

not brought us into the land \hspace{2em}or \hspace{2em} not built the Temple for us:

it would have been enough for us.
\end{english}
\end{center}

 
 %%%%%%% PLAGUES CODA
 
 עַל אַחַת, כַּמָּה וְכַמָּה, טוֹבָה כְפוּלָה וּמְכֻפֶּלֶת לַמָּקוֹם עָלֵינוּ: שֶׁהוֹצִיאָנוּ מִמִּצְרַיִם, וְעָשָׂה בָהֶם שְׁפָטִים, וְעָשָׂה בֵאלֹהֵיהֶם, וְהָרַג אֶת בְּכוֹרֵיהֶם, וְנָתַן לָנוּ אֶת מָמוֹנָם, וְקָרַע לָנוּ אֶת הַיָּם, וְהֶעֱבִירָנוּ בְתוֹכוֹ בֶּחָרָבָה, וְשִׁקַּע צָרֵנוּ בְּתוֹכוֹ, וְסִפֵּק צָרְכֵּנוּ בַּמִּדְבָּר אַרְבָּעִים שָׁנָה, וְהֶאֱכִילָנוּ אֶת הַמָּן, וְנָתַן לָנוּ אֶת הַשַׁבָּת, וְקֵרְבָנוּ לִפְנֵי הַר סִינַי, וְנָתַן לָנוּ אֶת הַתּוֹרָה, וְהִכְנִיסָנוּ לְאֶרֶץ יִשְׂרָאֵל, וּבָנָה לָנוּ אֶת בֵּית הַבְּחִירָה לְכַפֵּר עַל כָּל עֲוֹנוֹתֵינוּ.
 
 \begin{english}
If for one, then how much more has God doubled and redoubled goodness for us: in taking us out of Egypt, judging them, dispatching their gods, killing their firstborns, giving us their wealth, splitting the sea for us, bringing us through it on dry land, drowning our tormentors in it, providing for our needs in the desert for forty years, feeding us the mana, giving us Shabos, bringing us close at Mount Sinai, giving us the Torah, bringing us into the land of Yisroel, and building for us the Temple to atone for our sins.
\end{english}
 
 %%%%%%% RABAN GAMLIEL
 
\vspace{3em}
 
\begin{center}
{\large \bfseries \textcolor{light-gray}{
פֶּסַח, מַצָה, וּמָרוֹר.
}}
\end{center}
 
 רַבָּן גַּמְלִיאֵל הָיָה אוֹמֵר: כָּל שֶׁלֹּא אָמַר שְׁלשָׁה דְּבָרִים אֵלּוּ בַּפֶּסַח, לֹא יָצָא יְדֵי חוֹבָתוֹ, וְאֵלּוּ הֵן: פֶּסַח, מַצָה, וּמָרוֹר.
 
 \begin{english}
Raban Gamliel would say, ``Anyone who does not speak of three things on Pesah has not fulfilled their obligation. They are the pesah, matso, and maror."
 \end{english}

\vspace{1em} 
 
\instruction{
יזהר שלא להגביה את הזרוע.
}{Be careful not to pick up the bone.}

פֶּסַח שֶׁהָיוּ אֲבוֹתֵינוּ אוֹכְלִים בִּזְמַן שֶׁבֵּית הַמִּקְדָּשׁ הָיָה קַיָּם, עַל שׁוּם מָה? עַל שׁוּם שֶׁפָּסַח הַקָּדוֹשׁ בָּרוּךְ הוּא עַל בָּתֵּי אֲבוֹתֵינוּ בְּמִצְרַיִם, שֶׁנֶּאֱמַר:
\pasuk{
וַאֲמַרְתֶּם זֶבַח פֶּסַח הוּא לַיי, אֲשֶׁר פָּסַח עַל בָּתֵּי בְנֵי יִשְׂרָאֵל בְּמִצְרַיִם בְּנָגְפּוֹ אֶת מִצְרַיִם, וְאֶת בָּתֵּינוּ הִצִּיל, וַיִּקֹּד הָעָם וַיִּשְּׁתַּחֲווּ.
}

\begin{english}
Pesah—that our ancestors ate at the time when the Beis Hamikdash was standing—what for? Because the Holy Blessed One passed over the homes of our ancestors in Egypt, as it says, \bibverse{And you shall say to them, this is the pesah offering to Hashem, who passed over the homes of the Yisroelites in Egypt when God struck the Egyptians, but our homes God saved. And the people bowed and prostrated.}
\end{english}

\vspace{1em} 

\instruction{
מראה את המצות למסבים ואומר:
}{Indicate the matso to all those present, and say:}

מַצָּה זוֹ שֶׁאָנוּ אוֹכְלִים, עַל שׁוּם מָה? עַל שׁוּם שֶׁלֹא הִסְפִּיק בְּצֵקָם שֶׁל אֲבוֹתֵינוּ לְהַחֲמִיץ עַד שֶׁנִּגְלָה עֲלֵיהֶם מֶלֶךְ מַלְכֵי הַמְּלָכִים, הַקָּדוֹשׁ בָּרוּךְ הוּא, וּגְאָלָם, שֶׁנֶּאֱמַר:
\pasuk{
וַיֹּאפוּ אֶת הַבָּצֵק אֲשֶׁר הוֹצִיאוּ מִמִּצְרַיִם עֻגֹת מַצּוֹת, כִּי לֹא חָמֵץ, כִּי גֹרְשׁוּ מִמִּצְרַיִם וְלֹא יָכְלוּ לְהִתְמַהְמֵהַּ, וְגַּם צֵדָה לֹא עָשׂו לָהֶם.
}

\begin{english}
This unleavened bread which we eat, what for? Because there was no break for our ancestors to leaven bread when the Ruler of Rulers—the Holy Blessed One—was revealed to them and redeemed them, as it says, \bibverse{And they baked the dough that they brought out of Egypt into unleavened cakes, since it did not rise, since they fled from Egypt and could not delay. Furthermore, they had not prepared other provisions.}
\end{english}

\vspace{1em} 

\instruction{
מראה את המרור למסבים ואומר:
}{Indicate the maror to all those present, and say:}

מָרוֹר זֶה שֶׁאָנוּ אוֹכְלִים, עַל שׁוּם מָה? עַל שׁוּם שֶׁמֵּרְרוּ הַמִּצְרִים אֶת חַיֵּי אֲבוֹתֵינוּ בְּמִצְרַיִם, שֶׁנֶּאֱמַר:
\pasuk{
וַיְמָרֲרוּ אֶת חַיֵּיהֶם בַּעֲבֹדָה קָשָה, בְּחֹמֶר וּבִלְבֵנִים וּבְכָל עֲבֹדָה בַּשָּׂדֶה אֶת כָּל עֲבֹדָתָם אֲשֶׁר עָבְדוּ בָהֶם בְּפָרֶךְ.
}

\begin{english}
This bitter herb which we eat, what for? Because the Egyptians embittered the lives of our ancestors in Egypt, as it says, \bibverse{And the Egyptians embittered their lives with hard work, with brick and mortar, and with all sorts of work in the field, and all the work that they were made to do was crushing work.}
\end{english}

\break

\vspace{1em}

בְּכָל דּוֹר וָדוֹר חַיָּב אָדָם לִרְאוֹת אֶת עַצְמוֹ כְּאִלּוּ הוּא יָצָא מִמִּצְרַיִם, שֶׁנֶּאֱמַר: 
\pasuk{
וְהִגַּדְתָּ לְבִנְךָ בַּיוֹם הַהוּא לֵאמֹר, בַּעֲבוּר זֶה עָשָׂה יי לִי בְּצֵאתִי מִמִּצְרָיִם.
}
לֹא אֶת אֲבוֹתֵינוּ בִּלְבָד גָּאַל הַקָּדוֹשׁ בָּרוּךְ הוּא, אֶלָּא אַף אוֹתָנוּ גָּאַל עִמָּהֶם, שֶׁנֶּאֱמַר: 
\pasuk{
וְאוֹתָנוּ הוֹצִיא מִשָׁם, לְמַעַן הָבִיא אֹתָנוּ, לָתֶת לָנוּ אֶת הָאָרֶץ אֲשֶׁר נִשְׁבַּע לַאֲבֹתֵנוּ.
}

\begin{english}
In each and every generation, a person is required to see themselves as though they had personally gone out of Egypt, as it says, \bibverse{And tell your child on that day saying, ``This is on account of what Hashem did for me when I went out of Egypt.} Not only my ancestors did the Holy Blessed One redeem, but we were also redeemed with them, as it says, \bibverse{And we were brought out of there, to bring us and give to us the land God promised to our ancestors.}
\end{english}

לְפִיכָךְ אֲנַחְנוּ חַיָּבִים לְהוֹדוֹת, לְהַלֵּל, לְשַׁבֵּחַ, לְפָאֵר, לְרוֹמֵם, לְהַדֵּר, לְבָרֵךְ, לְעַלֵּה וּלְקַלֵּס לְמִי שֶׁעָשָׂה לַאֲבוֹתֵינוּ וְלָנוּ אֶת כָּל הַנִסִּים הָאֵלוּ: הוֹצִיאָנוּ מֵעַבְדוּת לְחֵרוּת מִיָּגוֹן לְשִׂמְחָה, וּמֵאֵבֶל לְיוֹם טוֹב, וּמֵאֲפֵלָה לְאוֹר גָּדוֹל, וּמִשִּׁעְבּוּד לִגְאֻלָּה. וְנֹאמַר לְפָנָיו שִׁירָה חֲדָשָׁה: הַלְלוּיָהּ.

\begin{english}
Because of this, we are required to thank, to praise, to exalt, to adorn, to raise up, to bless, to elevate, and to extoll the One who made all these miracles for us and for our ancestors; who brought us from slavery to freedom, from sorrow to joy, from mourning to goodness, from deepest darkness to splendid light, from service to redemption. And therefore let us sing a new song, {\scshape Halleluyah}.
\end{english}

\begin{center}
{\glossfont

הַלְלוּ יָהּ הַלְלוּ עַבְדֵי יי הַלְלוּ אֶת שֵׁם יי.

יְהִי שֵׁם יי מְבֹרָךְ מֵעַתָּה וְעַד עוֹלָם.

מִמִּזְרַח שֶׁמֶשׁ עַד מְבוֹאוֹ מְהֻלָּל שֵׁם יי.

רָם עַל כָּל גּוֹיִם יי עַל הַשָּׁמַיִם כְּבוֹדוֹ.

מִי כַּיי אֱלֹהֵינוּ הַמַּגְבִּיהִי לָשָׁבֶת.

הַמַּשְׁפִּילִי לִרְאוֹת בַּשָּׁמַיִם וּבָאָרֶץ. 

מְקִימִי מֵעָפָר דָּל מֵאַשְׁפֹּת יָרִים אֶבְיוֹן. 

לְהוֹשִׁיבִי עִם נְדִיבִים עִם נְדִיבֵי עַמּוֹ.

מוֹשִׁיבִי עֲקֶרֶת הַבַּיִת אֵם הַבָּנִים שְׂמֵחָה הַלְלוּיָהּ:

\vspace{1em}

בְּצֵאת יִשְׂרָאֵל מִמִּצְרָיִם בֵּית יַעֲקֹב מֵעַם לֹעֵז.

הָיְתָה יְהוּדָה לְקָדְשׁוֹ יִשְׂרָאֵל מַמְשְׁלוֹתָיו.

הַיָּם רָאָה וַיָּנֹס הַיַּרְדֵּן יִסֹּב לְאָחוֹר.

הֶהָרִים רָקְדוּ כְאֵילִים גְּבָעוֹת כִּבְנֵי צֹאן.

מַה לְּךָ הַיָּם כִּי תָנוּס הַיַּרְדֵּן תִּסֹּב לְאָחוֹר.

הֶהָרִים תִּרְקְדוּ כְאֵילִים גְּבָעוֹת כִּבְנֵי צֹאן.

מִלִּפְנֵי אָדוֹן חוּלִי אָרֶץ מִלִּפְנֵי אֱלוֹהַּ יַעֲקֹב.

הַהֹפְכִי הַצּוּר אֲגַם מָיִם חַלָּמִישׁ לְמַעְיְנוֹ מָיִם:
}
\end{center}

\break

%%%%%% KOS SHENI

\begin{center}
{\large \bfseries \textcolor{light-gray}{
כוֹס שֶנִי
}}
\end{center}

\instruction{
מגביהים את הכוס עד גאל ישראל.
}{Pick up the cup for the whole blessing.}

בָּרוּךְ אַתָּה יי אֱלֹהֵינוּ מֶלֶךְ הָעוֹלָם, אֲשֶׁר גְּאָלָנוּ וְגָאַל אֶת אֲבוֹתֵינוּ מִמִּצְרַיִם, וְהִגִּיעָנוּ לַלַּיְלָה הַזֶּה לֶאֱכָל בּוֹ מַצָּה וּמָרוֹר. כֵּן יי אֱלֹהֵינוּ וֵאלֹהֵי אֲבוֹתֵינוּ יַגִּיעֵנוּ לְמוֹעֲדִים וְלִרְגָלִים אֲחֵרִים הַבָּאִים לִקְרָאתֵנוּ לְשָׁלוֹם, שְׂמֵחִים בְּבִנְיַן עִירֶךָ וְשָׂשִׂים בַּעֲבוֹדָתֶךָ. וְנֹאכַל שָׁם מִן הַזְּבָחִים וּמִן הַפְּסָחִים אֲשֶׁר יַגִּיעַ דָּמָם עַל קִיר מִזְבַּחֲךָ לְרָצוֹן, וְנוֹדֶה לְךָ שִׁיר חָדָש עַל גְּאֻלָּתֵנוּ וְעַל פְּדוּת נַפְשֵׁנוּ. בָּרוּךְ אַתָּה יי גָּאַל יִשְׂרָאֵל.

בָּרוּךְ אַתָּה יי אֱלֹהֵינוּ מֶלֶךְ הָעוֹלָם בּוֹרֵא פְּרִי הַגָּפֶן.

\begin{english}
Blessed are you, Hashem, our God, ruler of the universe, who redeemed us and redeemed our ancestors from Egypt, and who brought us to this night to eat on it matso and maror. So too may Hashem, our God, and our ancestor's God will bring us to other appointed times and festivals that are coming to decree peace for us, to gladden us in the rebuilding of your city, and for us to rejoice in your service. And let us eat there from the offerings and the pesahim whose blood will be brought on the horns of your altar as you wish, and let us praise you with a new song on your deliverance of us and on your redemption of our lives' breaths. Blessed are you, Hashem, who redeemed Israel.

Blessed are you, Hashem, our God, ruler of the universe, who created the fruit of the vine.
\end{english}

\begin{center}
\instruction{
שותין את הכוס בהסבת שמאל.
}{Drink the second cup reclining.}
\end{center} 