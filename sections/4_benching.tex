\siman{
בָּרֵךְ
}

\begin{center}
{\large \bfseries \textcolor{light-gray}{
כוֹס שְלִישִי
}}
\end{center}

\instruction{
שלשה שאכלו כאחד חיבין לזמן והמזמן פותח על הכוס:
}{If three ate together, one of them leads by inviting those present to bless God. The leader should pour (but not yet dilute) the third cup of wine and hold it in their right hand until it is blessed and drunk.}

\pasuk{
שִׁיר הַמַּעֲלוֹת בְּשׁוּב יי אֶת שִׁיבַת צִיּוֹן הָיִינוּ כְּחֹלְמִים. אָז יִמָּלֵא שְׂחוֹק פִּינוּ וּלְשׁוֹנֵנוּ רִנָּה אָז יֹאמְרוּ בַגּוֹיִם הִגְדִּיל יי לַעֲשׂוֹת עִם אֵלֶּה. הִגְדִּיל יי לַעֲשׂוֹת עִמָּנוּ הָיִינוּ שְׂמֵחִים. שׁוּבָה יי אֶת שְׁבִיתֵנוּ כַּאֲפִיקִים בַּנֶּגֶב. הַזֹּרְעִים בְּדִמְעָה בְּרִנָּה יִקְצֹרוּ. הָלוֹךְ יֵלֵךְ וּבָכֹה נֹשֵׂא מֶשֶׁךְ הַזָּרַע בֹּא יָבוֹא בְרִנָּה נֹשֵׂא אֲלֻמֹּתָיו.
}

\vspace{1em}

\instruction{}{The leader reads the following lines, with the others present answering with the lines in grey.}

% ZIMUN

\begin{center}
הַב לַן וְנֵיבְרֵך!

\vspace{1em}
\colorbox{light-gray}{
יְהִי שֵׁם יְיָ מְבֹרָךְ מֵעַתָּה וְעַד עוֹלָם.
}\vspace{1em}

\instruction{}{If ten are present, say {\instructionfont \texthebrew{אלהינו}}.}


יְהִי שֵׁם יְיָ מְבֹרָךְ מֵעַתָּה וְעַד עוֹלָם. בִּרְשׁוּת: 

\begin{tabular}{c c c c}
\textenglish{{\itshape the hosts}}
		& \textenglish{\itshape the priests}
			& \textenglish{\itshape honored guests} \\
בַּעַל(י) וּבַעַל(ו)ת בֵיתָא הַדֵין & וְכֹהָנָן & וְמְרָנָן וְרַבָּנָן וְרַבּוֹתַי \\
\end{tabular}

נְבָרֵךְ (אֱלהֵינוּ) שֶׁאָכַלְנוּ מִשֶּׁלוֹ.

\vspace{1em}
\colorbox{light-gray}{
בָּרוּךְ (אֱלהֵינוּ) שֶׁאָכַלְנוּ מִשֶּׁלוֹ וּבְטוּבוֹ חָיִינוּ.
}\vspace{1em}

בָּרוּךְ (אֱלהֵינוּ) שֶׁאָכַלְנוּ מִשֶּׁלוֹ וּבְטוּבוֹ חָיִינוּ.

\end{center}

\break

% BENCHING

בָּרוּךְ אַתָּה יי אֱלֹהֵינוּ מֶלֶךְ הָעוֹלָם, הַזָּן אֶת הָעוֹלָם כֻּלּוֹ בְּטוּבוֹ בְּחֵן בְּחֶסֶד וּבְרַחֲמִים, הוּא נֹתֵן לֶחֶם לְכָל-בָּשָׂר כִּי לְעוֹלָם חַסְדּוֹ, וּבְטוּבוֹ הַגָּדוֹל תָּמִיד לֹא חָסַר לָנוּ וְאַל יֶחְסַר לָנוּ מָזוֹן לְעוֹלָם וָעֶד, בַּעֲבוּר שְׁמוֹ הַגָּדוֹל, כִּי הוּא אֵל זָן וּמְפַרְנֵס לַכֹּל, וּמֵטִיב לַכֹּל וּמֵכִין מָזוֹן לְכָל-בִּרְיּוֹתָיו אֲשֶׁר בָּרָא. 

\begin{center}
{\large \bfseries
בָּרוּךְ אַתָּה יי. הַזָּן אֶת הַכּל:
}
\end{center}

\instruction{}{The leader dilutes the cup of wine before continuing on to the next blessing.}

\vspace{1em}

נוֹדֶה לְּךָ יי אֱלֹהֵינוּ עַל שֶׁהִנְחַלְתָּ לַאֲבוֹתֵינוּ אֶרֶץ חֶמְדָּה טוֹבָה וּרְחָבָה, וְעַל שֶׁהוֹצֵאתָנוּ יי אֱלֹהֵינוּ מֵאֶרֶץ מִצְרַיִם וּפְדִיתָנוּ מִבֵּית עֲבָדִים, וְעַל בְּרִיתְךָ שֶׁחָתַמְתָּ בִּבְשָׂרֵנוּ וְעַל תּוֹרָתְךָ שֶׁלִּמַּדְתָּנוּ וְעַל חֻקֶּיךָ שֶׁהוֹדַעְתָּנוּ, וְעַל חַיִּים חֵן וָחֶסֶד שֶׁחוֹנַנְתָּנוּ, וְעַל אֲכִילַת מָזוֹן שָׁאַתָּה זָן וּמְפַרְנֵס אוֹתָנוּ תָּמִיד, בְּכָל יוֹם וּבְכָל עֵת וּבְכָל שָׁעָה. וְעַל הַכֹּל יי אֱלֹהֵינוּ אֲנַחְנוּ מוֹדִים לָךְ וּמְבָרְכִים אוֹתָךְ, יִתְבָּרַךְ שִׁמְךָ בְּפִי כָּל חַי תָּמִיד לְעוֹלָם וָעֶד, כַּכָּתוּב: 
\pasuk{
וְאָכַלְתָּ וְשָׂבַעְתָּ, וּבֵרַכְתָּ אֶת יי אֱלֹהֶיךָ עַל הָאָרֶץ הַטּוֹבָה אֲשֶּׁר נָתַן לָךְ. 
}

\begin{center}
{\large \bfseries
בָּרוּךְ אַתָּה יי, עַל הָאָרֶץ וְעַל הַמָזוֹן.
}
\end{center}

\vspace{1em}

\begin{minipage}{0.48\linewidth}
\hebstruction{
בשאר ימים
}\hfill\textenglish{\LR{\itshape On other days}}

רַחֶם יי אֱלֹהֵינוּ עַל יִשְׂרָאֵל עַמֶּךָ, וְעַל יְרוּשָׁלַיִם עִירֶךָ, וְעַל צִיּוֹן מִשְׁכַּן כְּבוֹדֶךָ, וְעַל מַלְכוּת בֵּית דָּוִד מְשִׁיחֶךָ, וְעַל הַבַּיִת הַגָדוֹל וְהַקָדוֹשׁ שֶׁנִּקְרָא שִׁמְךָ עָלָיו.
\end{minipage}\hspace{0.01\linewidth}\vrule{}\hspace{0.01\linewidth}
\begin{minipage}{0.48\linewidth}
\hebstruction{
בשבת
}\hfill\textenglish{\LR{\itshape On Shabos}}

נַחֲמֵנוּ בִּיְרוּשָׁלַיִם עִירֶךָ, וּבְצִיּוֹן מִשְׁכַּן כְּבוֹדֶךָ, וּבְמַלְכוּת בֵּית דָּוִד מְשִׁיחֶךָ, וּבַבַּיִת הַגָדוֹל וְהַקָדוֹשׁ שֶׁנִּקְרָא שִׁמְךָ עָלָיו.
\end{minipage}

אֱלֹהֵינוּ, אָבִינוּ, רְעֵנוּ, זוּנֵנוּ, פַרְנְסֵנוּ וְכַלְכְּלֵנוּ וְהַרְוִיחֵנוּ, וְהַרְוַח לָנוּ יי אֱלֹהֵינוּ מְהֵרָה מִכָּל צָרוֹתֵינוּ. וְנָא אַל תַּצְרִיכֵנוּ יי אֱלֹהֵינוּ, לֹא לִידֵי מַתְּנַת בָּשָׂר וָדָם וְלֹא לִידֵי הַלְוָאָתָם, כִּי אִם לְיָדְךָ הַמְּלֵאָה הַפְּתוּחָה הַקְּדוֹשָׁה וְהָרְחָבָה, שֶׁלֹא נֵבוֹשׁ וְלֹא נִכָּלֵם לְעוֹלָם וָעֶד.

\begin{framed}
\instruction{
בשבת מוסיפין:
}{On Shabos, add:}

רְצֵה וְהַחֲלִיצֵנוּ יי אֱלֹהֵינוּ בְּמִצְוֹתֶיךָ וּבְמִצְוַת יוֹם הַשְׁבִיעִי הַשַׁבָּת הַגָּדוֹל וְהַקָדוֹשׂ הַזֶּה. כִּי יוֹם זֶה גָּדוֹל וְקָדוֹשׁ הוּא לְפָנֶיךָ לִשְׁבָּת בּוֹ וְלָנוּחַ בּוֹ בְּאַהֲבָה כְּמִצְוַת רְצוֹנֶךָ. וּבִרְצוֹנְךָ הָנִיחַ לָנוּ יי אֱלֹהֵינוּ שֶׁלֹּא תְהֵא צָרָה וְיָגוֹן וַאֲנָחָה בְּיוֹם מְנוּחָתֵנוּ.
\end{framed}

\vspace{1em}

אֱלֹהֵינוּ וֵאלֹהֵי אֲבוֹתֵינוּ, יַעֲלֶה וְיָבֹא וְיַגִּיעַ, וְיֵרָאֶה וְיֵרָצֶה וְיִשָּׁמַע, וְיִפָּקֵד וְיִזָּכֵר זִכְרוֹנֵנוּ וּפִקְדוֹנֵנוּ וְזִכְרוֹן אֲבוֹתֵינוּ, וְזִכְרוֹן מָשִׁיחַ בֶּן דָּוִד עַבְדֶּךָ, וְזִכְרוֹן יְרוּשָׁלַיִם עִיר קָדְשֶׁךָ, וְזִכְרוֹן כָּל עַמְּךָ בֵּית יִשְׂרָאֵל לְפָנֶיךָ לִפְלֵטָה, לְטוֹבָה, לְחֵן וּלְחֶסֶד וּלְרַחֲמִים, לְחַיִּים נוסח ספרד: טוֹבִים וּלְשָׁלוֹם, בְּיוֹם חַג הַמַּצּוֹת הַזֶה. זָכְרֵנוּ יי אֱלֹהֵינוּ בּוֹ לְטוֹבָה, וּפָּקְדֵנוּ בוֹ לִבְרָכָה, וְהוֹשִׁיעֵנוּ בוֹ לְחַיִּים נוסח ספרד: טוֹבִים ; וּבּדְבַר יְשׁוּעָה וְרַחֲמִים חוּס וְחָנֵּנוּ, וְרַחֵם עָלֵינוּ וְהוֹשִׁיעֵנוּ, כִּי אֵלֶיךָ עֵינֵינוּ, כִּי אֵל מֶלֶךְ חַנּוּן וְרַחוּם אָתָּה.

\vspace{1em}

\begin{minipage}{0.30\linewidth}
\hebstruction{
בשאר ימים
}\hfill\textenglish{\LR{\itshape Otherwise}}
וּבְנֵה יְרוּשָׁלַיִם עִיר הַקֹּדֶשׁ בִּמְהֵרָה בְיָמֵינוּ. 
\begin{center}
{\large \bfseries
בָּרוּךְ אַתָּה יי, בּוֹנֵה בְרַחֲמָיו יְרוּשָׁלָיִם.
}\end{center}
\end{minipage}\hspace{0.01\linewidth}\vrule{}\hspace{0.01\linewidth}
\begin{minipage}{0.66\linewidth}
\hebstruction{
בשבת
}\hfill\textenglish{\LR{\itshape On Shabos}}

וְהַרְאֵנוּ יי אֱלֹהֵינוּ בְּנֶחָמַת צִיוֹן עִירֶךָ וּבְבִנְיַן יְרוּשָׁלַיִם עִיר קָדְשֶׁךָ כִּי אַתָּה הוּא בַּעַל הַיְשׁוּעוֹת וּבַעַל הַנֶּחָמוֹת. 

\begin{center}
{\large \bfseries
בָּרוּךְ אַתָּה יי, מְנַחֵם צִיּוֹן וּבוֹנֵה יְרוּשָׁלָיִם.
}\end{center}
\end{minipage}

\vspace{1em}

בָּרוּךְ אַתָּה יי, אֱלֹהֵינוּ מֶלֶךְ הָעוֹלָם, הָאֵל אָבִינוּ, מַלְכֵּנוּ, אַדִירֵנוּ, בּוֹרְאֵנוּ, גֹּאֲלֵנוּ, יוֹצְרֵנוּ, קְדוֹשֵׁנוּ קְדוֹשׁ יַעֲקֹב, רוֹעֵנוּ רוֹעֵה יִשְׂרָאַל, הַמֶּלֶךְ הַטּוֹב וְהַמֵּטִיב לַכֹּל, שֶׁבְּכָל יוֹם וָיוֹם הוּא הֵטִיב, הוּא מֵטִיב, הוּא יֵיטִיב לָנוּ. הוּא גְמָלָנוּ הוּא גוֹמְלֵנוּ הוּא יִגְמְלֵנוּ לָעַד, לְחֵן וּלְחֶסֶד וּלְרַחֲמִים וּלְרֶוַח הַצָּלָה וְהַצְלָחָה, בְּרָכָה וִישׁוּעָה נֶחָמָה פַּרְנָסָה וְכַלְכָּלָה, וְרַחֲמִים וְחַיִּים וְשָׁלוֹם וְכָל טוֹב; וּמִכָּל טוּב לְעוֹלָם עַל יְחַסְּרֵנוּ.

\vspace{1em}

\begin{center}

{\bf הָרַחֲמָן}
הוּא יִמְלוֹךְ עָלֵינוּ לְעוֹלָם וָעֶד.

{\bf הָרַחֲמָן}
הוּא יִתְבָּרַךְ בַּשָּׁמַיִם וּבָאָרֶץ.

{\bf הָרַחֲמָן}
הוּא יִשְׁתַּבַּח לְדוֹר דּוֹרִים, וְיִתְפָּאַר בָּנוּ לָעַד וּלְנֵצַח נְצָחִים, וְיִתְהַדַּר בָּנוּ לָעַד וּלְעוֹלְמֵי עוֹלָמִים.

{\bf הָרַחֲמָן}
הוּא יְפַרְנְסֵנוּ בְּכָבוֹד.

{\bf הָרַחֲמָן}
הוּא יִשְׁבּוֹר עֻלֵּנוּ מֵעַל צַּוָּארֵנוּ, וְהוּא יוֹלִיכֵנוּ קוֹמְמִיוּת לְאַרְצֵנוּ.

{\bf הָרַחֲמָן}
הוּא יִשְׁלַח לָנוּ בְּרָכָה מְרֻבָּה בַּבַּיִת הַזֶּה, וְעַל שֻׁלְחָן זֶה שֶׁאָכַלְנוּ עָלָיו.

{\bf הָרַחֲמָן}
הוּא יִשְׁלַח לָנוּ אֶת אֵלִיָּהוּ הַנָּבִיא זָכוּר לַטּוֹב, וִיבַשֵּׂר לָנוּ בְּשׂוֹרוֹת טוֹבוֹת יְשׁוּעוֹת וְנֶחָמוֹת.



\break

{\large \bf
הָרַחֲמָן
}
הוּא יְבָרֵךְ 
\hrule

\instruction{}{In one's own home, say:}

\begin{tabular}{c c c c c}

	& \textenglish{\itshape my father(s)}
		& \textenglish{\itshape my mother(s)}
			& \textenglish{\itshape my spouse}
				& \textenglish{\itshape my children}\\
					
אוֹתִי & וְאָבי מוֹרי & וְאִמּי מוֹרָתי & וְאֶת אִשִי/אִשְׁתִּי & וְאֶת זַרְעִי \\
\end{tabular}

וְאֶת בֵּיתִי וְאֶת כָּל אֲשֶׁר לִי,

\vspace{0.1em}
\hrule
\vspace{0.1em}

\instruction{}{In the home of another, say:}

\begin{tabular}{c c}
\textenglish{\itshape (my father) the host(s)}
	& \textenglish{\itshape (my mother) the hostess(es)}\\

אֶת (אָבִי מוֹרִי) בַּעַל(י) הַבַּיִת הַזֶּה & וְאֶת (אִמִּי מוֹרָתִי) בַּעֲלַ(ו)ת הַבַּיִת הַזֶּה \\
\end{tabular}

\begin{tabular}{c c c}
	& \textenglish{\itshape their children}
		& \\
אוֹתָם וְאֶת בֵּיתָם & וְאֶת זַרְעָם & וְאֶת כָּל אֲשֶׁר לָהֶם, \\
\end{tabular}

\vspace{0.1em}
\hrule
\vspace{0.1em}

\instruction{}{At a communal meal, say:}

אֶת כָל הַמְסוּבִּין כַּאן,

\vspace{0.1em}
\hrule
\vspace{0.1em}

\instruction{}{All continue:}

אוֹתָנוּ וְאֶת כָּל אֲשֶׁר לָנוּ, כְּמוֹ שֶׁנִּתְבָּרְכוּ אֲבוֹתֵינוּ אַבְרָהָם יִצְחָק וְיַעֲקֹב
\glossed{
בַּכֹּל, מִכֹּל, כֹּל,
}
כֵּן יְבָרֵךְ אוֹתָנוּ כֻּלָּנוּ יַחַד בִּבְרָכָה שְׁלֵמָה. וְנֹאמַר: אָמֵן.

\end{center}

\begin{framed}
\instruction{
יש מברכין הבעל הבית בברכה זו:
}{Guests recite the following blessing for their hosts:}

\begin{minipage}[t]{0.47\linewidth}
\instruction{}{For grammatically masculine host(s)}

יְהִי רָצוֹן, שֶׁלֹּא יֵבוֹשׁ(וּ) בַּעַל(י) הַבַּיִת בָּעוֹלָם הַזֶּה, וְלֹא יִכָּלֵמ(וּ) לָעוֹלָם הַבָּא, וְיִצְלַח(וּ) מְאֹד בְּכָל נְכָסָיו(הֶם), וְיִהְיוּ נְכָסָיו(הֶם) מֻצְלָחִים וּקְרוֹבִים לָעִיר, וְאַל יִשְׁלֹט שָׂטָן לֹא בְּמַעֲשֵׂי יָדָיו(הֶם) וְלֹא בְּמַעֲשֵׂי יָדֵינוּ, וְאַל יִזְדַקֵק לֹא לְפָנָיו(הֶם) וְלֹא לְפָנֵינוּ שׁוּם דְבַר הַרְהוֹר חֵטְא וַעֲבֵרָה וְעָוֹן מֵעַתָּה וְעַד עוֹלָם.
\end{minipage}\hspace{0.01\linewidth}\vrule{}\hspace{0.01\linewidth}
\begin{minipage}[t]{0.47\linewidth}
\instruction{}{For grammatically feminine host(s)}

יְהִי רָצוֹן, שֶׁלֹּא תֵבוֹשׁ(נָה) בַּעַל הַבַּיִת בָּעוֹלָם הַזֶּה, וְלֹא תִכָּלֵמ(נָה) לָעוֹלָם הַבָּא, וְתִצְלַח(נָה) מְאֹד בְּכָל נְכָסָיו(הֶם), וְיִהְיוּ נְכָסָיו(הֶן) וּנְכָסֵינוּ מֻצְלָחִים וּקְרוֹבִים לָעִיר, וְאַל יִשְׁלֹט שָׂטָן לֹא בְּמַעֲשֵׂי יָדָיו(הֶן) וְלֹא בְּמַעֲשֵׂי יָדֵינוּ, וְאַל יִזְדַקֵק לֹא לְפָנָיו(הֶן) וְלֹא לְפָנֵינוּ שׁוּם דְבַר הַרְהוֹר חֵטְא וַעֲבֵרָה וְעָוֹן מֵעַתָּה וְעַד עוֹלָם.
\end{minipage}
\end{framed}

בַּמָרוֹם יְלַמְּדוּ עֲלֵיהֶם וְעָלֵינוּ זְכוּת שֶׁתְּהֵא לְמִשְׁמֶרֶת שָׁלוֹם. וְנִשָׂא בְרָכָה מֵאֵת יי, וּצְדָקָה מֵאלֹהֵי יִשְׁעֵנוּ, וְנִמְצָא חֵן וְשֵׂכֶל טוֹב בְּעֵינֵי אֱלֹהִים וְאָדָם.


\begin{center}

\begin{framed}
\instruction{
בשבת מוסיפין:
}{On Shabos, add:}

{\bf הָרַחֲמָן}
הוּא יַנְחִילֵנוּ יוֹם שֶׁכֻּלוֹ שַׁבָּת וּמְנוּחָה לְחַיֵּי הָעוֹלָמִים.
\end{framed}

{\bf הָרַחֲמָן}
הוּא יַנְחִילֵנוּ יוֹם שֶׁכֻּלוֹ טוֹב, יוֹם שֶׁכֻּלּוֹ אָרוֹךְ, יוֹם שֶׁצַּדִּיקִים יוֹשְׁבִים וְעַטְרוֹתֵיהֶם בְּרָאשֵׁיהֶם וְנֶהֱנִין מִזִּיו הַשְּׁכִינָה, וִיהִי חֶלְקֵנוּ עִמָּהֶם.

{\bf הָרַחֲמָן}
הוּא יְזַכֵּנוּ לִימוֹת הַמָּשִׁיחַ וּלְחַיֵּי הָעוֹלָם הַבָּא.
\end{center}

\vspace{1em}

מִגְדּוֹל יְשׁוּעוֹת מַלְכּוֹ, וְעֹשֶׂה חֶסֶד לִמְשִׁיחוֹ, לְדָוִד וּלְזַרְעוֹ עַד עוֹלָם. עֹשֶׂה שָׁלוֹם בִּמְרוֹמָיו, הוּא יַעֲשֶׂה שָׁלוֹם עָלֵינוּ וְעַל כָּל יִשְׂרָאַל. וְאִמְרוּ: אָמֵן.

יְראוּ אֶת יי קְדֹשָׁיו, כִּי אֵין מַחְסוֹר לִירֵאָיו. כְּפִירִים רָשׁוּ וְרָעֵבוּ, וְדֹרְשֵׁי יי לֹא יַחְסְרוּ כָל טוֹב. הוֹדוּ לַיי כִּי טוֹב, כִּי לְעוֹלָם חַסְדּוֹ. פּוֹתֵחַ אֶת יָדֶךָ, וּמַשְׂבִּיעַ לְכָל חַי רָצוֹן. בָּרוּךְ הַגֶּבֶר אֲשֶׁר יִבְטַח בַּיי, וְהָיָה יי מִבְטַחוֹ. נַעַר הָיִיתִי גַם זָקַנְתִּי, וְלֹא רָאִיתִי צַדִיק נֶעֱזָב, וְזַרְעוֹ מְבַקֶּשׁ לָחֶם. יי עֹז לְעַמוֹ יִתֵּן, יי יְבָרֵךְ אֶת עַמוֹ בַשָׁלוֹם.

\vspace{1em}

\instruction{
מוגזין כוס שלישי, ואומר קודם הברכה:
}{Raise the third cup, and bless:}

הִנְנִי מוּכָן וּמְזֻמָּן לְקַיֵם מִצְוַת כּוֹס שְלִישִי שֶהוּא כְּנֶגֶד בְּשׂורַת הַיְשׁוּעָה, שֶׁאָמַר הַקָּדוֹשׁ בָּרוּךְ הוּא לְיִשְׂרָאֵל 
\pasuk{
וְגָאַלְתִּי אֶתְכֶם בִּזְרוֹעַ נְטוּיָה וּבִשְׁפָטִים גְדוֹלִים.
}

בָּרוּךְ אַתָּה יי אֱלֹהֵינוּ מֶלֶךְ הָעוֹלָם בּוֹרֵא פְּרִי הַגָפֶן.

\instruction{
שותין בהסבת שמאל.
}{Drink, reclining on the left side. The cup should be passed to all present, to drink if they wish.}

\break